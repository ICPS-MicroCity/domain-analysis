\section{Contesto}
\textit{Micro City} concerns areas with bounded temporal and spatial extension and with a high concentration of people.
Inside the \textit{Micro City}, different types of activity may be carried out: they might be services (offering experiences or products) or events. Services are always operative or available, while events may have a limited duration.
Most of the people in the \textit{Micro City} are strongly interested in the available activities during the lifetime of the \textit{Micro City}. This period may vary depending on the specific context of the \textit{Micro City}. It defines temporal bounds for activities.
Thus, the fundamental elements of the \textit{Micro City} are:
\begin{itemize}
    \item Is assumed that all guests are endowed with a wearable device (like a smartphone), so that they can interact with activities.
    \item A \textit{Micro City} is limited in space: it presents well-defined physical bounds, that specify domain's limits.
    \item A \textit{Micro City} is limited in time: it presents well-defined operativity periods, that specify when guests can benefit from activities.
    \item A \textit{Micro City} presents heterogeneous activities that are physically situated and distributed inside it. These activities justify the existence of the \textit{Micro City} because guests are interested in them. Activities may be static or dynamic. Static activities cannot physically move, while dynamic ones may move if necessary. Is assumed that activities can gather and send interesting information to guests. Activities may satisfy a certain amount of guests with a certain frequency. Therefore, the number of guests that can benefit from an activity simultaneously is limited.
    \item The guests that take part in the \textit{Micro City} may be individuals or group of people. Guests change, partially or completely, periodically. Is assumed that guests are highly interested in the activities. Moreover, is assumed that groups of guests have similar interests and, therefore, they move together inside the \textit{Micro City}. Also, is assumed that a group of guests uses a single wearable device in order to benefit from activities. A \textit{Micro City} also presents internal operators, distinguished from guests, that do not benefit from activities (because they manage them).
    \item The high amount of guests that attend activities may cause the prolongation of waiting time before benefitting from them. This also causes the formation of queues.
    \item Guests may pay a certain amount of money (fee) in order to access the \textit{Micro City} and/or to benefit from activities.
    \item Activities may be proactively recommended to guests, depending on their wearable-tracked position.
    \item If guests accept recommendations, they may receive rewards. Rewards vary depending on the context of the \textit{Micro City} and they may concern:
        \begin{itemize}
            \item A discount that may concern a particular product or can be used freely.
            \item A cashback that may promote sustainable actions and behaviours.
            \item Accumulable points that allow to collect prizes offered by the \textit{Micro City}.
            \item The improvement of an experience, such as the reduction of waiting time in a queue.
        \end{itemize}
    \item A \textit{Micro City} presents a map that may bring along routes that link different activities. Guests can identify activities using the map.
\end{itemize}