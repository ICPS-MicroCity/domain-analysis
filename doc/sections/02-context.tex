\section{Contesto}
Il contesto della \textit{MicroCity} riguarda zone con estensione spaziale limitata e all'interno delle quali vi è una alta concentrazione di persone per un determinato lasso di tempo.
All'interno della \textit{MicroCity} si svolgono delle attività che possono essere: servizi che forniscono esperienze, prestazioni o prodotti, oppure eventi. Nel primo caso questi sono sempre operativi o disponibili mentre gli eventi hanno una durata ben definita.
La maggior parte delle persone presenti nell’ambiente sono interessate alle attività che la zona offre in quel lasso di tempo. Questo periodo varia a seconda del contesto specifico della \textit{MicroCity} e definisce i limiti di operatività delle attività al suo interno.
Si possono individuare i seguenti elementi fondamentali:
\begin{itemize}
    \item Si assume che tutti i cittadini siano dotati di un dispositivo \textit{wearable} come uno \textit{smartphone} in modo che possano interagire con le attività della \textit{MicroCity}.
    \item Sono limitate nello spazio, ovvero presentano dei confini fisici ben delineati al di fuori dei quali si esce dal dominio della \textit{MicroCity}.
    \item Sono limitate nel tempo: ciò significa che è possibile individuare dei periodi di operatività ben definiti al di fuori dei quali le attività all'interno non sono più fruibili dagli ospiti.
    \item Sono dotate di attività eterogenee, fisicamente distribuite e situate al loro interno. Tali attività giustificano l'esistenza stessa della \textit{MicroCity} e sono di interesse per gli ospiti. Le attività possono essere statiche o dinamiche: nel primo caso, una volta fissate non possono essere riposizionate; nel secondo invece possono spostarsi dove è necessario o utile. Si assume che queste riescano a raccogliere e comunicare informazioni di interesse per gli ospiti e/o relative al proprio stato.
    Le attività soddisfano un certo numero di ospiti con una certa frequenza temporale. Di conseguenza, il numero di ospiti che possono fruire dell'attività contemporaneamente è limitato.
    \item Gli ospiti che prendono parte alla \textit{MicroCity} sono: singole persone o gruppi di persone; questi variano periodicamente, in parte o completamente. Gli ospiti sono tutti considerati incentivati a partecipare alle attività.
    Si assume che i gruppi di ospiti abbiano interessi simili e pertanto si muovono insieme nella \textit{MicroCity}. Inoltre, si assume che un gruppo di ospiti utilizzi un unico \textit{wearable} per partecipare alle attività della \textit{MicroCity}.
    Inoltre, sono presenti anche operatori interni, distinti dagli ospiti, che non prendono parte alle attività ma le dirigono.
    \item La moltitudine di ospiti che frequentano in massa le attività può provocare il prolungarsi dei tempi di attesa per fruire delle attività con formazione di code.
    \item \'E possibile che gli ospiti debbano pagare una somma di denaro per accedere alla MicroCity e/o per fruire delle attività.
    \item Possono essere opportunisticamente consigliate  attività agli ospiti in cambio di un \textit{reward}. Essi variano a seconda del contesto della \textit{MicroCity} e possono prevedere:
    \begin{itemize}
        \item Uno sconto se l'attività in questione vende prodotti o servizi a sè; tale sconto può riguardare un prodotto in particolare oppure può essere fruibile in qualsiasi modo all'interno di un periodo di validità.
        \item Un tipo di cashback se l'attività in questione promuove azioni e comportamenti sostenibili e vuole incentivare i micro-cittadini a perseguirle.
        \item Dei punti cumulabili che permettono di riscuotere dei premi forniti dalla \textit{MicroCity} stessa.
        \item Il miglioramento di un' esperienza:
        come ad esempio la riduzione dei tempi di attesa per svolgere un'attività.
    \end{itemize}
    \item Presentano una mappa, che induce a possibili percorsi che collegano le diverse attività; queste ultime possono essere identificate dagli ospiti facendo uso della mappa.
\end{itemize}
